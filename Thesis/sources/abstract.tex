\thispagestyle{plain}
\begin{abstract}

In this MSc. thesis, we present a numerical framework in Julia for solving problems in the emerging field of Waveguide Quantum Electrodynamics (WQED). The framework is based on collision quantum optics, where a localized quantum system interacts with a collection of time-bins one at a time. This physically intuitive picture enables researchers familiar with QUTiP in Python, QuantumToolbox in Matlab, or QuantumOptics.jl in Julia to set up waveguide QED simulations with relative ease. Despite the conceptually simple picture, we demonstrate that the framework is capable of tackling complex Waveguide QED problems. These include the scattering of one or two-photon pulses on emitters or cavities, internal coupling between waveguides leading to the prediction of Fano resonances, and also non-Markovian systems where emitted light is reflected back and leads to feedback mechanisms. 

\

Note that this framework is a contribution to the community in the form of a well-documented open-source project through the package "WaveguideQED.jl" in Julia. During the development, various contributions were also made to the "QuantumOptics.jl" package in Julia, which are now merged into the main library through the pull request: \href{https://github.com/qojulia/QuantumOpticsBase.jl/pull/86}{https://github.com/qojulia/QuantumOpticsBase.jl/pull/86}.

Documentation of the code can be found here: \href{https://qojulia.github.io/WaveguideQED.jl/dev/}{https://qojulia.github.io/WaveguideQED.jl/dev/}.
\end{abstract}